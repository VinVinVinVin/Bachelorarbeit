\chapter{Myoelektrische Steuerung}
Zur Steuerung von Oberarm-Prothesen und in jüngster Zeit auch -Orthesen werden oft elektromyographische (EMG) Signale benutzt. Diese Steuerungsmethode, auch myoelektrische Steuerung genannt, findet bereits seit über 30 Jahren in klinischen Anwendungen Bedeutung \citep{Parker.2006}. Bei dieser Methode werden elektrische Signale, die bei bewussten Bewegungen entstehen, genutzt, um das Gerät zu steuern. Dieses Kapitel wird einen Überblick über die Funktionsweise von EMG-Signalen und wie diese genutzt werden können, um eine Orthese zu steuern.

\section{Elektromyographische Signale}
EMG Signale sind geringe Spannungen, die kurz vor der bewussten Kontraktion von Muskeln generiert werden. Dabei werden bei bewussten Bewegungen Aktionspotentiale vom Gehirn oder zentralen Nervensystem an die motorischen Einheiten gesendet, welche sich daraufhin zusammenziehen.  Diese elektrische Aktivität der Muskeln kann entweder durch Nadelelektroden, welche unter die Haut in den Muskel eingeführt werden, oder über Oberflächenelektroden, welche direkt auf der Haut über dem betreffenden Muskel angebracht werden gemessen werden. Diese Oberflächenmessung ist sowohl in der Diagnostik als auch in der Anwendung verbreitet, da sie nicht-invasiv abläuft und wenig Risiken für den Patienten mitbringt.\\

Die Amplitude eines EMG-Signals hängt dabei von mehreren Faktoren ab und liegt im $\mu$V bis unterem mV-Bereich \citep{Luca.2006}. Dabei hängen Amplitude und Frequenz des Signals unter anderem mit folgenden Faktoren zusammen \citep{Gerdle.2013}:
\begin{itemize}
	\item Zeitpunkt und Stärke der Muskelbewegung
	\item Kontakt zwischen Elektrode und Hautoberfläche
	\item Entfernung zwischen Muskel und Elektrode
	\item Material zwischen Muskel und Elektrode (z.B. Haut- und Fettgewebe)
	\item Elektroden- und Verstärkerqualität
\end{itemize}
Dabei ist nur der erste Faktor, der Zeitpunkt und die Intensität der Muskelbewegung von Interesse für die Messung, die anderen Faktoren sollten möglichst wenig Einfluss auf das Messergebnis haben. Dies kann z.B. dadurch realisiert werden, dass bei jedem Versuch baugleiche Elektroden verwendet werden oder die Elektroden möglichst genau an der gleichen Stelle angebracht werden. Ziel sollte dabei ein möglichst hohes Verhältnis von Signal zu Rauschen (Signal to Noise Ratio, SNR) bei gleichzeitig möglichst geringer Varianz sein. \\

Eine wichtige Methode zum entfernen von unerwünschtem, statischen Rauschen ist die Bipolar-Recording-Technique, bei der zwei Elektroden benutzt werden und die Differenz der beiden Signalen gebildet wird \citep{Day.2000}. Hierdurch können Störungen, die bei beiden Elektroden auftreten wie z.B. Hintergrundrauschen durch A/C-Stromversorgung entfernt werden. Wird jedoch ein Muskel in der Nähe einer der beiden Elektroden angeregt, so ist das Signal bei der näheren der beiden Elektroden viel stärker als bei der anderen, das Signal wird also nicht entfernt.\\

Der zweite wichtige Faktor für die Qualität der EMG-Messung ist die Impedanz zwischen Haut und Elektroden. Dank moderner Vorverstärker ist es nicht mehr wichtig, wie groß die Impedanz an den beiden Elektorden ist, sondern eine zeitliche Stabilität und Balance zwischen den beiden Elektroden \citep{Hermens.1999}. Da eine bipolare Messung nur Signalanteile, die bei beiden Elektroden gleich Auftreten entfernen kann, benötigt diese Methode möglichst ähnliche Bedinungen an beiden Elektroden. Aus dem selben Grund ist auch eine zeitliche Stabilität von hoher Wichtigkeit für eine verlässliche Messung. \\

Ein weiter Qualitätsfaktor ist das Überlagern von mehreren EMG-Signalen verschiedener Muskeln, auch Cross Talk genannt. Die Oberfläche-Elekroden messen naturgemäß nicht einen expliziten Muskel, sondern das elektrische Potential an der Hautoberfläche. Da die Stärke eines EMG-Signals proportional zu der Muskelgröße ist, können größere Muskeln Kleine überdecken. Dazu sinkt die Signalstärke exponentiell mit der Entfernung von der Elektrode, weshalb es schwer ist, tiefer liegende Muskeln zu messen. Diese Probleme können jedoch durch eine korrekte Elektrodengröße und -positionierung gelöst werden. \\

In welchem Zusammenhang ein EMG-Signal mit der resultierenden Muskelkraft steht ist noch nicht abschließend erforscht, wobei eine Vielzahl von Autoren eine Nicht-Lineare Proportionalität vermuten \citep{Karlsson.2001,Gregor.2002,Bilodeau.2003}. Dies hängt auch mit den oben erläuterten Problemen der EMG-Messung zusammen, weshalb es schwer ist, den Anteil einzelner Muskeln an dem Gesamtsignal zu bestimmen. Theoretische Analysen haben ergeben, dass die Amplitude des Signals bei isometrischen Kontraktionen mit der Quadratwurzel der generierten Kraft steigen sollte \citep{Uliam.2012,Basmajian.1985}. \\

\section{Signalverarbeitung}

TODO: Benötigte Filter etc. 

\section{Myoelektrische Steuerung}
Nach der Merkmalsextraktion wird eine Steuerungmechanismus benötigt, der basierend auf dem aufgenommenen Signal die Antriebseinheit steuert. Da es sich hierbei um einen Open-Loop-Kreis, also keine Rückkopplung stattfindet, handelt, spricht man von einer Steuerung, keiner Regelung. Abhängig vom erzielten Signal bieten sich hier drei Steuermechanismen an: Ein binärer Treshold-Mechanismus, bei der die Antriebseinheit mit einem fixen Drehmoment betrieben wird, sobald eine gewisse Schwelle übertreten wurde, ein Stepper-Mechanismus, bei dem Treshold-Mechanismen kombiniert werden, um mehrere diskrete Drehmomente zu ermöglichen. Oder eine proportionale Steuerung, bei der das Drehmoment direkt proportional zum Signal steigt. Welcher dieser drei Mechanismen realisierbar ist, hängt vorwiegend von der Qualität des aufgenommenen Signals ab. \\

Vorteile einer myoelektrischen Steuerung im Vergleich zu Joystick oder Drucksensoren sind unter andrem die Möglichkeit, die mechanische Unterstützung eines Exoskeletts problemlos einzustellen \citep{Ferris.2007},  das Signal durch Oberflächenelektroden nicht-invasiv aufgenommen werden kann und die benötigte Muskelrestaktivität verhältnismäßig klein ist \citep{Parker.2006} und einer intakten Gliedmaße ähnelt. Zusätzlich kann die Qualität der Steuerung durch neue Signalverarbeitungs-Algorithmen oder verbesserte Elektroden einfach und ohne medizinische Eingriffe verbessert werden und das Produkt somit langfristig im Einsatz bleiben. Auf der anderen Seite benötigt das System eine gewisse Einarbeitungszeit durch Patient und Therapeuten, bis die entsprechenden Parameter zur Steuerung angepasst sind. Da das aufgegriffene EMG-Signal stark von der Positionierung der Elektroden abhängt ist es ebenfalls wichtig, diese Position möglichst konstant zu halten. Ein weiterer Nachteil liegt darin, dass eine myoelektrische Steuerung naturgemäß nur möglich ist, wenn EMG-Signale vorliegen. Dadurch ist ein Einsatz bei einigen Krankheiten, z.B. Querschnittslähmung, nicht möglich. \\

