%%%%%%%%%%%%%%%%%%%%%%%%%%%%%%%%%%%%%%%%%%%%%%%%%%%%%%%%%%%%%%%%%%%%%%%%%%%%%%%%%%%%%%%%%%%%%%%%%%%%%%%%
%	MY LAB BOOK - TUM Neuroelectronics group
%%%%%%%%%%%%%%%%%%%%%%%%%%%%%%%%%%%%%%%%%%%%%%%%%%%%%%%%%%%%%%%%%%%%%%%%%%%%%%%%%%%%%%%%%%%%%%%%%%%%%%%%

% Author: Philipp Rinklin
% Template by: Leroy Grob & Philipp Rinklin

% Version 2.0
%
% Changelist v2:
% -----------------------------------------
% - Separated main file into a file containing all packages,
%   a file containing all formatting specific things and the main
%   file containing document structure.

%%%%%%%%%%%%%%%%%%%%%%%%%%%%%%%%%%%%%%%%%%%%%%%%%%%%%%%%%%%%%%%%%%%%%%%%%%%%%%%%%%%%%%%%%%%%%%%%%%%%%%%%
%Set document type:
\documentclass[10pt,a4paper,final,hyperref,oneside]{labbook}
%%%%%%%%%%%%%%%%%%%%%%%%%%%%%%%%%%%%%%%%%%%%%%%%%%%%%%%%%%%%%%%%%%%%%%%%%%%%%%%%%%%%%%%%%%%%%%%%%%%%%%%%
%	General information
%%%%%%%%%%%%%%%%%%%%%%%%%%%%%%%%%%%%%%%%%%%%%%%%%%%%%%%%%%%%%%%%%%%%%%%%%%%%%%%%%%%%%%%%%%%%%%%%%%%%%%%%
\def\initials{%
		N.H.}
\def\fullname{%
		Nico Hertel}
\def\worktype{%
		Integration einer Antriebseinheit in einer Orthese zur gezielten Kraftunterst\"utzung }
\def\timespan{%
		06.03.2017 - 07.04.2017}

%%%%%%%%%%%%%%%%%%%%%%%%%%%%%%%%%%%%%%%%%%%%%%%%%%%%%%%%%%%%%%%%%%%%%%%%%%%%%%%%%%%%%%%%%%%%%%%%%%%%%%%%
%Import packages from package list.
%%%%%%%%%%%%%%%%%%%%%%%%%%%%%%%%%%%%%%%%%%%%%%%%%%%%%%%%%%%%%%%%%%%%%%%%%%%%%%%%%%%%%%%%%%%%%%%%%%%%%%%%
%%%%%%%%%%%%%%%%%%%%%%%%%%%%%%%%%%%%%%%%%%%%%%%%%%%%%%%%%%%%%%%%%%%%%%%%%%%%%%%%%%%%%%%%%%%%%%%%%%%%%%%%
%Packages for LabBook template:
%%%%%%%%%%%%%%%%%%%%%%%%%%%%%%%%%%%%%%%%%%%%%%%%%%%%%%%%%%%%%%%%%%%%%%%%%%%%%%%%%%%%%%%%%%%%%%%%%%%%%%%%
%
% Version: 02
%
% Changelist v02:
% ---------------------------------------------------------
% - FontSpec package removed, main font changed to 'Arial'.
%   Does not have to be compiled with XeLaTeX anymore.
% - Minor changes to comments.
% - Addition of svg package.
% - tabularx package removed since column width can be con-
%   trolled by adding "@{\extracolsep{\fill}}" before defining
%		columns (see example table for details).
%%%%%%%%%%%%%%%%%%%%%%%%%%%%%%%%%%%%%%%%%%%%%%%%%%%%%%%%%%%%%%%%%%%%%%%%%%%%%%%%%%%%%%%%%%%%%%%%%%%%%%%%

%Packages for character encoding and hyphenation.
\usepackage[T1]{fontenc}
\usepackage[english]{babel}

%Packages for graphics import.
\usepackage{graphicx,svg}

%Caption package to control the typesetting of float captions.
\usepackage[
labelfont={bf},
font={small},
singlelinecheck=false,
format=plain
]{caption}

%Standard AMS math package for math formatting
\usepackage{amsmath,amssymb,wasysym}

%SIunitx package for formatting of SI units. For options, see doc. Custom units are included.
\usepackage{siunitx}
\AtBeginDocument{\sisetup{per-mode=reciprocal, list-final-separator={, and }, range-phrase={\text{--}}, sticky-per, number-unit-product=\text{\,}, product-units=power, list-units=single, range-units=single, quotient-mode=fraction, fraction-function=\nicefrac, retain-unity-mantissa=false,math-rm=\mathrm, text-rm=\rmfamily, detect-weight, binary-units=true}}
\DeclareSIUnit\molar{M}
\DeclareSIUnit\particles{particles}
\DeclareSIUnit\rpm{rpm}
\DeclareSIUnit\units{\textsc{u}}
\DeclareSIUnit\ccm{ccm}
\DeclareSIUnit\psi{psi}
\DeclareSIUnit\rad{rad}
\DeclareSIUnit\fps{fps}
\DeclareSIUnit\sq{\ensuremath{\Box}}

%Macro to define function for times. I.e. \hms{1} gives '1 h', \hms{2;30} gives '2 h 30 min', \hms{;;30} gives '30 sec', etc...
\ExplSyntaxOn
\NewDocumentCommand \hms { o > { \SplitArgument { 2 } { ; } } m }
  {
    \group_begin:
      \IfNoValueF {#1}
        { \keys_set:nn { siunitx } {#1} }
      \siunitx_hms_output:nnn #2
    \group_end:
  }
\cs_new_protected:Npn \siunitx_hms_output:nnn #1#2#3
  {
    \IfNoValueF {#1}
      {
        \tl_if_blank:nF {#1}
          {
            \SI {#1} { \hour }
            \IfNoValueF {#2} { ~ }
          }
      }
    \IfNoValueF {#2}
      {
        \tl_if_blank:nF {#2}
          {
            \SI {#2} { \minute }
            \IfNoValueF {#3} { ~ }
          }
      }
    \IfNoValueF {#3}
      { \tl_if_blank:nF {#3} { \SI {#3} { \second } } }
  }
\ExplSyntaxOff

%Setspace package for line spacing macros.
\usepackage{setspace}

%Booktabs package for horizontal rules (\toprule, \midrule, \bottomrule).
\usepackage{booktabs}

%Scrlayer-scrpage package for formatting of header and footer.
\usepackage[headsepline=true,plainheadsepline=true,%
						footsepline=true,plainfootsepline=true]%
					 {scrlayer-scrpage}

%Package to input chemical formulae easier (\ce{H20}, \ce{H2S04}, \ce{Ca^2+}, etc.).
\usepackage[version=3]{mhchem}

%For hyperlinks between chapters, citations, etc.
\usepackage[plainpages=false,hidelinks]{hyperref}

%mathpazo package to change math font.
\usepackage{mathpazo}

%enumitem for control over list formatting.
\usepackage{enumitem}

%xcolor package for definition of colors etc.
\usepackage{xcolor}

%Code formatting
\usepackage{listings} %Library to list the code

%Tikz package for drawing line graphs etc.
\usepackage{tikz}
\usetikzlibrary{positioning} %library for relative alignments

%Circuitikz package for drawing circuits.
\usepackage{xstring}
\usepackage[europeanresistor]{circuitikz}

%grffile package helps with special characters in file and directory names.
\usepackage{grffile}

%geometry package allows flexible formatting of page margins etc.
\usepackage[vcentering]{geometry}

%lipsum package for blind text.
\usepackage{lipsum}

%Define a command to read in all .tex files from the individual years.
\def\inputLabDays{%
	\immediate\write18{cmd /c FileList_LabDays.bat}%
  \InputIfFileExists{./OutList.tmp}{}}

%Define a command to read in all .tex files from the SOP directory.
\def\inputSOPs{%
	\immediate\write18{cmd /c FileList_SOPs.bat}%
  \InputIfFileExists{./SOPs/OutList.tmp}{}}
	
%Scrhack package to fix \float@addtolist warning.
\usepackage{scrhack}

\usepackage[utf8]{inputenc}
\usepackage{natbib}
\usepackage{pdfpages}

%%%%%%%%%%%%%%%%%%%%%%%%%%%%%%%%%%%%%%%%%%%%%%%%%%%%%%%%%%%%%%%%%%%%%%%%%%%%%%%%%%%%%%%%%%%%%%%%%%%%%%%%
%Set up page format (mainly margins and header & footer)
%%%%%%%%%%%%%%%%%%%%%%%%%%%%%%%%%%%%%%%%%%%%%%%%%%%%%%%%%%%%%%%%%%%%%%%%%%%%%%%%%%%%%%%%%%%%%%%%%%%%%%%%

%Set up page margins.
\geometry{left=20mm, top=20mm, right=20mm, bottom=40mm}

%Set up header.
%\cohead[\includegraphics{header.JPG}]{\includegraphics{header.JPG}}
%\setlength{\headheight}{323}

%Define TUM blau for use in footer.
\definecolor{TUMblau}{rgb}{0,0.396,0.741}

%Set up footer.
\renewcommand{\footfont}{\normalfont}
\lofoot[\color{TUMblau}\initials]{\color{TUMblau}\initials}
\cofoot[\color{TUMblau}\thepage]{\color{TUMblau}\thepage}
\rofoot[\color{TUMblau}\timespan]{\color{TUMblau}\timespan}

%%%%%%%%%%%%%%%%%%%%%%%%%%%%%%%%%%%%%%%%%%%%%%%%%%%%%%%%%%%%%%%%%%%%%%%%%%%%%%%%%%%%%%%%%%%%%%%%%%%%%%%%
%Miscellaneous formatting tweaks.
%%%%%%%%%%%%%%%%%%%%%%%%%%%%%%%%%%%%%%%%%%%%%%%%%%%%%%%%%%%%%%%%%%%%%%%%%%%%%%%%%%%%%%%%%%%%%%%%%%%%%%%%

%Change footnote indentation.
\deffootnote[1em]{1em}{1em}{\textsuperscript{\thefootnotemark}}

%Set default sans-serif to phv ('Arial') and change default family to sans-serif.
\renewcommand{\sfdefault}{phv}
\renewcommand{\familydefault}{\sfdefault}

%Set line spacing to be 1.5.
\onehalfspacing

%Don't allow widows and orphans for text and display equations.
\clubpenalty = 10000
\widowpenalty = 10000
\displaywidowpenalty = 10000

%%%%%%%%%%%%%%%%%%%%%%%%%%%%%%%%%%%%%%%%%%%%%%%%%%%%%%%%%%%%%%%%%%%%%%%%%%%%%%%%%%%%%%%%%%%%%%%%%%%%%%%%
%Definition of a global table style (see example entry for usage).
%%%%%%%%%%%%%%%%%%%%%%%%%%%%%%%%%%%%%%%%%%%%%%%%%%%%%%%%%%%%%%%%%%%%%%%%%%%%%%%%%%%%%%%%%%%%%%%%%%%%%%%%

%Define a command to globally control setting of column headers.
\newcommand*\heading{}

%Define which rules should be applied in the top, after the header, between normal lines and in the bottom.
\newcommand*\beforeheading{\toprule}
\newcommand*\afterheading{\midrule}
\newcommand*\normalline{}
\newcommand*\lastline{\bottomrule}

%%%%%%%%%%%%%%%%%%%%%%%%%%%%%%%%%%%%%%%%%%%%%%%%%%%%%%%%%%%%%%%%%%%%%%%%%%%%%%%%%%%%%%%%%%%%%%%%%%%%%%%%
%Definition of how code snippets should be set (delete if not needed).
%%%%%%%%%%%%%%%%%%%%%%%%%%%%%%%%%%%%%%%%%%%%%%%%%%%%%%%%%%%%%%%%%%%%%%%%%%%%%%%%%%%%%%%%%%%%%%%%%%%%%%%%

\definecolor{mygreen}{rgb}{0,0.6,0}
\definecolor{mygray}{rgb}{0.5,0.5,0.5}
\definecolor{mymauve}{rgb}{0.58,0,0.82}

\lstset{
  backgroundcolor=\color{white},   % choose the background color; you must add \usepackage{color} or \usepackage{xcolor}
  basicstyle=\footnotesize,        % the size of the fonts that are used for the code
  breakatwhitespace=false,         % sets if automatic breaks should only happen at whitespace
  breaklines=true,                 % sets automatic line breaking
  captionpos=b,                    % sets the caption-position to bottom
  commentstyle=\color{mygreen},    % comment style
  deletekeywords={...},            % if you want to delete keywords from the given language
  escapeinside={\%*}{*)},          % if you want to add LaTeX within your code
  extendedchars=true,              % lets you use non-ASCII characters; for 8-bits encodings only, does not work with UTF-8
  frame=false,                     % adds a frame around the code
  keepspaces=true,                 % keeps spaces in text, useful for keeping indentation of code (possibly needs columns=flexible)
  keywordstyle=\color{blue},       % keyword style
  language=Octave,                 % the language of the code
  otherkeywords={*,...},           % if you want to add more keywords to the set
  numbers=left,                    % where to put the line-numbers; possible values are (none, left, right)
  numbersep=5pt,                   % how far the line-numbers are from the code
  numberstyle=\tiny\color{mygray}, % the style that is used for the line-numbers
  rulecolor=\color{black},         % if not set, the frame-color may be changed on line-breaks within not-black text (e.g. comments (green here))
  showspaces=false,                % show spaces everywhere adding particular underscores; it overrides 'showstringspaces'
  showstringspaces=false,          % underline spaces within strings only
  showtabs=false,                  % show tabs within strings adding particular underscores
  stepnumber=2,                    % the step between two line-numbers. If it's 1, each line will be numbered
  stringstyle=\color{mymauve},     % string literal style
  tabsize=2,                       % sets default tabsize to 2 spaces
  title=\lstname                   % show the filename of files included with \lstinputlisting; also try caption instead of title
}

%%%%%%%%%%%%%%%%%%%%%%%%%%%%%%%%%%%%%%%%%%%%%%%%%%%%%%%%%%%%%%%%%%%%%%%%%%%%%%%%%%%%%%%%%%%%%%%%%%%%%%%%
%-------------------------------------------------------------------------------------------------------
\begin{document}
%-------------------------------------------------------------------------------------------------------
%%%%%%%%%%%%%%%%%%%%%%%%%%%%%%%%%%%%%%%%%%%%%%%%%%%%%%%%%%%%%%%%%%%%%%%%%%%%%%%%%%%%%%%%%%%%%%%%%%%%%%%%
%Create title page.
%%%%%%%%%%%%%%%%%%%%%%%%%%%%%%%%%%%%%%%%%%%%%%%%%%%%%%%%%%%%%%%%%%%%%%%%%%%%%%%%%%%%%%%%%%%%%%%%%%%%%%%%

%\begin{titlepage}
%\thispagestyle{scrheadings}
%\vspace*{6cm}
%\centering
%\textcolor{TUMblau}{\Huge\bfseries Bachelorarbeit\\
%\vspace{5mm}
%\LARGE\bfseries \emph{\worktype}}\\
%\vspace{10mm}
%{\large \fullname}\\
%Lehrstuhl f\"ur Mirkotechnik und Medizinger\"atetechnik\\
%Fakult\"at f\"ur Maschinenwesen
%\end{titlepage}
\includepdf{./Kapitel/Deckblatt.pdf}

%%%%%%%%%%%%%%%%%%%%%%%%%%%%%%%%%%%%%%%%%%%%%%%%%%%%%%%%%%%%%%%%%%%%%%%%%%%%%%%%%%%%%%%%%%%%%%%%%%%%%%%%
%Create TOC
%%%%%%%%%%%%%%%%%%%%%%%%%%%%%%%%%%%%%%%%%%%%%%%%%%%%%%%%%%%%%%%%%%%%%%%%%%%%%%%%%%%%%%%%%%%%%%%%%%%%%%%%
\renewcommand{\contentsname}{Inhaltsverzeichnis}
\tableofcontents
\addtocontents{toc}{\protect\setcounter{tocdepth}{1}} 
\clearpage
																										
%%%%%%%%%%%%%%%%%%%%%%%%%%%%%%%%%%%%%%%%%%%%%%%%%%%%%%%%%%%%%%%%%%%%%%%%%%%%%%%%%%%%%%%%%%%%%%%%%%%%%%%%
%Add lab write ups
%%%%%%%%%%%%%%%%%%%%%%%%%%%%%%%%%%%%%%%%%%%%%%%%%%%%%%%%%%%%%%%%%%%%%%%%%%%%%%%%%%%%%%%%%%%%%%%%%%%%%%%%

%Include complete notes (comment out/in as needed).
%\inputLabDays

%If you wish to only review a specific lab day input
%the path of the .tex file here (comment out/in as needed).
%\bibliographystyle{agsm}
\part{Theorie}
\chapter{Begriffe}
Die in dieser Arbeit genutzten Begriffe haben in ihrem Alltagsgebrauch meist keine präzise Definition. Falls nicht anders angegeben sind die angegebenen Begriffe wie folgt zu verstehen:\\
\section{Orthesen}
Unter Orthesen ist eine "extern angebrachte Vorrichtung, die aus einzelnen Baugruppen besteht"\citep{Specht.2008} zu verstehen. Nach ISO Norm 8551:2003 erfüllt eine Orthese eine der folgenden Funktionen: Fehlstellungen vorbeugen, reduzieren oder halten, Gelenkbeweglichkeit begrenzen oder verbessern, L\"angenausgleich, schlaffe L\"ahmungen kompensieren, spastische L\"ahmungen kontrollieren oder die Belastung auf das Gewebe reduzieren/umverteilen. \\
\section{Angetriebenes Exoskelett}
Gem\"aß der amerikanischen Food and Drug Agency (FDA) beschreibt der Begriff Angetriebenes Exoskelett (Powered Exoskeleton) ein verschriebenes Ger\"at, das aus einer externen Orthese und einer Antriebseinheit besteht und für medizinische Zwecke an paralysierten oder geschw\"achten Gliedmaßen des Patienten angebracht wird\citep{FoodandDrugAdministration.2016}. 

\chapter{Stand der Technik}
Elektrisch oder mechanisch unterstützende Orthesen versprechen ihren Benutzern eine kürzere und umfassendere Genesung nach einem Herzinfarkt oder eine Verbesserung der Lebensbedingungen nach Rückenmarksverletzung oder anderen neurologischen Verletzungen\citep{Taub.1993,Hesse.2003}. Demnach versuchen eine Vielzahl von Herstellern, auf diesem Markt Fuß zu fassen und Produkte zu etablieren. Dieses Kapitel soll einen Überblick über die aktuell am Markt vorhandenen Produkte geben.\\
Diese Orthesen sind für den langfristigen Gebrauch ausgelegt und sollen Patienten miteingeschränkter Mobilität helfen, sich zu bewegen und mit der Umwelt zu interagieren. Entsprechend ihren Anforderungen verfügen diese Orthesen über eine portable Energieversorgung und sind begrenzt was Rechenkraft und Antriebstechnik angeht. \\


\section{Bionic Leg (AlterG inc.)}  
Die AlterG Bioneic Leg Orthese ist eine einseitige Knieorthese\citep{Bishop.2012}, die für Menschen mit einer asymetrischen Beeinträchtigung der Gehfähigkeit wie dem Brown-Séquard-Syndrom entworfen wurde. Gesteuert wird die Orthese über vier Kraft-Sensoren, die auf der Fußplatte fest installiert sind. Zusätzlich wird der Kniewinkel sowie die übertragene Kraft kontinuierlich während jeder Bewegung gemessen, um ein Überstrecken zu verhindern. Die Orthese kann beim Gehen, aufstehen und Treppensteigen unterstützen, muss jedoch mit Krücken benutzt werden. Ein Therapeut kann neben dem Bewegungsspielraum auch die Kraftunterstützung durch die Motoren sowie die Threshold-Kraft der Drucksensoren am Fuß einstellen. Die Batterie hat eine Lebensdauer von 2-3 Stunden, die gesamte Orthese wiegt ca. 3,6kg.

\section{Ekso Exoskeleton (Eksobionics Ltd.)} 
Die Orthese von Eksobionics besitzt steuerbare Knie- und Hüftgelenke sowie ein automatisches Sprunggelenk für beide Beine und ermöglicht eine Aufstehbewegung sowie ebenes Gehen\citep{Strausser.2011}. Die Orthese kann dabei in vier Modi betrieben werden: die Bewegung kann entweder durch den Patienten bzw. Therapeuten per Knopfdruck gesteuert werden oder auf Hüftbewegungen bzw. eine Gewichtsverlagerung reagieren und so auch in verschiedenen Phasen der Rehabilitation benutzt werden. Beim gehen werden zusätzlich spezielle Krücken benötigt, die mit einem Drucksensor am Fußende ausgestattet sind. Hierdurch kann sichegrestellt werden, dass beie Krücken fest am Boden stehen, bevor die Einheit sich in Bewegung setzt. 

\section{Indego (Parker Hannifin)}     
Die Indego-Orthese ist eine beidseitige Hüft-, Knie- und Fußgelenk-Orthese\citep{Murray.2014}, die aufgrund ihres dünnen Profils auch in einem Rollstuhl getragen werden kann. Duch Oberkörperbewegung und dadurch resultierende Gewichtsverlagerung unterstützt sie den Träger bei Aufsteh- und Hinsetz-Bewegungen sowie bei ebenen gehen\citep{Hartigan.2015}. Da die Orthese aus verschiedenen zusammensetzbaren Modulen besteht ist sie einfach und schnell anziehbar. Als Antriebseinheit wird dabei neben klassischen Motoren auch Elektronenstimulation (functional electrical stimulation, FES), bei der schwache elektrische Pulse die gelähmten Muskeln stimulieren und so zur Kontraktion anregen und somit bis zu 35\% der Leistung der Orthese ausmachen\citep{Ekelem.2015}. Durch diese Technik können die Muskeln einer querschnittsgelähmten Person weiterhin aktiv bleiben, was einige Vorteile für die Gesundheit des Patienten mitbringt\citep{Martin.2012}. So kann Muskelmasse erhalten bleiben\citep{Johnston.2005}, verlorene Knochenmasse wiederhergestellt werden\citep{Frotzler.2009}  oder die Sauerstoffaufnahme erhöht werden\citep{Bhambhani.2000}. Auch diese Konstruktion benötigt Gehhilfen wie Krücken. 
 
\section{Re-Walk (Argo Medical)}
Der israelische Re-Walk ist eine beidseitige Knie- und Hüftorthese, die mittel Knopfdruck und Gewichtsverlagerung gesteuert wird. Hierbei werden Aufsteh- und Hinsetzbewegungen sowie ebenes Gehen unterstützt. Auch Treppensteigen ist bei ausreichender Übung möglich. Die Gehgeschwindigkeit ist dabei stufenlos einstellbar und beträgt maximal 2.2 km/h\citep{Zeilig.2012}. Die Orthese kann nur Knie- und Hüftgelenk aktiv steuern, das Fußgelenk wird passiv über Federn gesteuert. Der Antrieb erfolgt über Elektromotoren. Die Orthese darf nur mit einer Gehhilfe betrieben werden. 

\section{HAL (Cyberdyne)}  
Das Hybrid Assistive Limb oder HAL (hybride untersützende Gliedmaße) Exoskelet aus Japan war das erste Exoskelet, das eine weltweite Sicherheitszerifizierung erhalten hat\citep{Sankai.2011,Suzuki.2007}. Das Exoskelet ist in zwei Versionen erhältlich, einmal als reine Beinorthese (HAL 3) sowie einmal als Ganzkörpersystem mit Arm-, Bein- und Torsoeinheit (HAL 5). Das System wurde sowohl zur Rehabilitation und zur Alltagsunterstützung als auch für schwere physische Arbeiten entworfen. So wurde eine modifizierte Version von HAL 5 bei der Nuklearkatastrophe von Fukushima eingesetzt, um Aufräumarbeiten zu verrichten\citep{LarryGreenemeier.2011}. Die Einheit unterstützt zwei Steuerungskonzepte, einen Voluntary-Control-Mode (CVC) und einen Autonomous-Control-Mode (CAC). Die Bewegung wird dabei im CVC-Modus über EMG Signale der Muskeln übernommen, es ist also eine Restmuskelaktivität benötigt\citep{Kubota.2013}. Dabei kann die Kraftunterstützung proportional zum gemessenen Signal für Ober- und Unterschenkel sowie Sprunggelenk getrennt gesteuert werden. Durch die myoelektrische Steuerung kann dieses Produkt besonders einfach durch den Benutzer erlernt werden, da nur eine gewöhnliche Bewegung initilaisiert werden muss. Sollte der Benutzer keine messbare Restmuskelaktivität mehr produzieren können, kann das HAL Exoskelet durch Druckplatten in speziellen Schuhen im CAC Modus geteuert werden. Die CE-Kennzeichnung (CE0197) erfordert  eine Gehhilfe. 

\section{Rex (RexBionics)}     
Rex ist eine beidseitige Hüft- und Knieorthese, die durch Elektromotoren betrieben wird. Rex ist dabei das einzige selbst-stabilisierende System auf dem Markt, die Orthese benötigt also keinerlei Gehhilfen wie Krücken oder einen Rollator\citep{Aach.2015,Lajeunesse.2016}. Das elektronische Balancesystem erfordert jedoch einen tiefen Schwerpunkt und ein hohes Gewicht von 38-40kg, ermöglicht aber Vorwärts- und Rückwärtsbewegungen, Treppen steigen sowie Stehen. Gesteuert wird die Einheit durch einen Joystick\citep{Barbareschi.2015}. 
\\
\begin{table}%
\begin{tabular}{p{2.5cm}|p{3.5cm}|p{2.5cm}|p{2.5cm}|p{1.8cm}|p{1.5cm}}
Name & Steuerung & Anwendergröße (cm) & Max. Anwendergewicht (kg) & Akkulaufzeit (h) & Eigengewicht (kg) \\
\hline
Rex & Joystick & 146-195 & 100 & 2 & 40 \\
Re-Walk & Gewichtsverlagerung & 150-190 & 100 & 3.5 & 20 \\
HAL & EMG & 150-196 & 100 & 1-2 & 17 \\
Indego & Gewichtsverlagerung + FES & 155-195 & 113 & 4 & 12 \\
Ekso & Gewichtsverlagerung & 150-190 & 100 & 4 & 23 \\

\end{tabular}
\caption{Merkmale der Unterkörper-Orthesen}
\label{Tabelle 1}
\end{table}
\\
Tabelle 1 vergleicht die verschiedenen Unterkörper-Orthesen auf Akkulaufzeit, Maximales Anwendergewicht, mögliche Körpergröße sowie Steuerungsmethode und Gewicht.

\section{MyoPro (Myomo)}
MyoPro ist eine aktive, nicht invasive Ellenbogen-Orthese die zur Therapie nach einem Schlaganfall entworfen wurde. Die gesamte Orthese wiegt unter 1,8kg und ist dank einer Akku-Einheit komplett mobil und im Alltag einsetzbar. Gesteuert wird die Orthese über EMG-Signale, die kontinuierlich an Bizeps und Trizeps aufgenommen werden und in einem PIC Microcontroller verarbeitet werden. Der Kontrollalgorithmus ermöglicht sowohl Strecken als auch Beugen des Arms. Dabei wird die Kraftunterstützung proportional zur Signalstärke gesteuert. Das EMG-System besteht aus vier electromyopraphischen Sensoren sowie analogen und digitalen Signalverarbeitungskomponenten \citep{Peters.2017}. Parameter wie der Verstärkungsfaktor der Muskelkraft (engl. Gain), Bewegungsfreiheit (engl. Degree of Motion) oder die Signalschwelle (Threshold) können vom behandelden Arzt individuell eingestellt werden. Die Bewegung der Orthese erfolgt übe 12V Gleichstrommotoren sowie zwei Stahlkabel welche bis zu 7Nm Drehmoment erzeugen \citep{Stein.2007}. MyoPro wird momentan nur in den USA vertrieben, soll jedoch in Zukunft über die deutsche Firmer Ottobock weltweit erhältlich sein.

\chapter{Stand der Forschung}
Während in der Industrie mobile Orthesen zur Erweiterung bzw. Wiederherstellung der Mobilität verbreitet sind, steht in der Forschung momentan öfter die Rehabilitation und Therapie im Vordergrund. Therapeutische Orthesen sind meist an ihren Ort gebunden, vorwiegend aufgrund der benötigten Energieversorgung bzw. durch die benötigte externe Ausstattung wie Kompressoren oder leistungsstarke Computer. Sie werden meist in Krankenhäusern oder Arztpraxen zur Therapie direkt nach einem Unfall oder Schlaganfall benutzt und sind nicht darauf ausgelegt, langfristig vom Patienten getragen zu werden. Mehrere Studen zeigen, dass nach einem Schlaganfall die Therapie im ersten Monat nach dem Anfall große Auswirkungen auf die Rehabitilation des Patienten hat \citep{Langhorne.2011, Prange.2006}. Dieses Kapitel stellt zwei Forschungsprojekte kurz vor. \\

\section{NEUROExos (BioRobotics Institute, Scuola Superiore Sant’Anna, Italien)}
Das Forschungsteam um Nicola Vitiello hat eine Ellenbogen Orthese zur Neurorehabilitation nach einem Schlaganfall entworfen \citep{Cempini.2013}. Die 35kg schwere Vorrichtung kann sowohl durch den Therapeuten als auch bei fortschreitendem Therapieerfolg durch den Patienten gesteuert werden. Durch eine innovative Verbindung der Ober- und Unterarmschienen richten sich die Drehachsen der Orthesen passiv an den Drehachsen des Ellenbogengelenks aus. Durch diesen Mechanismus können unerwünschte Parallelverschiebungskräfte minimiert werden \citep{Vitiello.2013}. Die Orthese unterstützt zwei Steuerungsmethoden: Robot-in-charge und patient-in-charge. Erstere wird vorwiegend zu Beginn der Behandlung benutzt, wenn der Patient seinen Arm noch nicht kontrolliert bewegen kann. Hier führt die Orthese eine Bewegung selbstständig aus, der Patient folgt dieser Bewegung passiv. Im Patient-in-Charge Modus dagegen liefert die Orthese nur unterstützende Arbeit, die eigentliche Bewegung wird durch den Patienten gesteuert. Die Muskelbewegung des Trägers wird über Feder und vier piezoresistive Dehnmessstreifen gemessen, die Position der Unterarm-Einheit kann durch zwei Potentiometer festgestellt werden.  Angetrieben wird die Orthese über eine Hydraulik und einen 1.1kV Drehstrommotor, welche ein Drehmoment von bis zu 30Nm ermöglichen \citep{Lenzi.2011}.\\
 
\section{Ankle-Foot-Orthese (Human Neuromechanics Laboratory, University of Michigan, USA)}
Diese Fußgelenksorthese wurde entworfen, um nach einem Schlaganfall das Wiedererlangen der Gehfähigkeit zu verbessern sowie um die Mechanik des menschlichen Ganges besser zu verstehen \citep{Ferris.2005}. Durch künstliche pneumatische Muskeln schafft die Orthese eine Kraftunterstützung von bis zu 70Nm zur Beugung des Fußgelenks \citep{Gordon.2007}. Angetrieben werden die Muskeln durch einen externen Elektromotor, weshalb die Orthese nur stationär betrieben werden kann. Die Orthese dient zur Kraftunterstützung, kann also nur eine Bewegung des Trägers verstärken. Hierfür wurde eine myoelektrische Steuerung realisiert. Mittels Elektromyografie (EMG) werden elektrische Signale an soleus und tibialis anterior gemessen, um die aufgebrachte Kraft proportional zu steuern \citep{Ferris.2017}. Die Orthese läst sich einfach innerhalb von einer Minute durch den Patienten oder einen Therapeuten anbringen und kann individuell an den Träger angepasst werden \citep{Ferris.2006}. 
\chapter{Myoelektrische Steuerung}
Zur Steuerung von Oberarm-Prothesen und in jüngster Zeit auch -Orthesen werden oft elektromyographische (EMG) Signale benutzt. Diese Steuerungsmethode, auch myoelektrische Steuerung genannt, findet bereits seit über 30 Jahren in klinischen Anwendungen Bedeutung \citep{Parker.2006}. Bei dieser Methode werden elektrische Signale, die bei bewussten Bewegungen entstehen, genutzt, um das Gerät zu steuern. Dieses Kapitel wird einen Überblick über die Funktionsweise von EMG-Signalen und wie diese genutzt werden können, um eine Orthese zu steuern.

\section{Elektromyographische Signale}
EMG Signale sind geringe Spannungen, die kurz vor der bewussten Kontraktion von Muskeln generiert werden. Dabei werden bei bewussten Bewegungen Aktionspotentiale vom Gehirn oder zentralen Nervensystem an die motorischen Einheiten gesendet, welche sich daraufhin zusammenziehen.  Diese elektrische Aktivität der Muskeln kann entweder durch Nadelelektroden, welche unter die Haut in den Muskel eingeführt werden, oder über Oberflächenelektroden, welche direkt auf der Haut über dem betreffenden Muskel angebracht werden gemessen werden. Diese Oberflächenmessung ist sowohl in der Diagnostik als auch in der Anwendung verbreitet, da sie nicht-invasiv abläuft und wenig Risiken für den Patienten mitbringt.\\

Die Amplitude eines EMG-Signals hängt dabei von mehreren Faktoren ab und liegt im $\mu$V bis unterem mV-Bereich \citep{Luca.2006}. Dabei hängen Amplitude und Frequenz des Signals unter anderem mit folgenden Faktoren zusammen \citep{Gerdle.2013}:
\begin{itemize}
	\item Zeitpunkt und Stärke der Muskelbewegung
	\item Kontakt zwischen Elektrode und Hautoberfläche
	\item Entfernung zwischen Muskel und Elektrode
	\item Material zwischen Muskel und Elektrode (z.B. Haut- und Fettgewebe)
	\item Elektroden- und Verstärkerqualität
\end{itemize}
Dabei ist nur der erste Faktor, der Zeitpunkt und die Intensität der Muskelbewegung von Interesse für die Messung, die anderen Faktoren sollten möglichst wenig Einfluss auf das Messergebnis haben. Dies kann z.B. dadurch realisiert werden, dass bei jedem Versuch baugleiche Elektroden verwendet werden oder die Elektroden möglichst genau an der gleichen Stelle angebracht werden. Ziel sollte dabei ein möglichst hohes Verhältnis von Signal zu Rauschen (Signal to Noise Ratio, SNR) bei gleichzeitig möglichst geringer Varianz sein. \\

Eine wichtige Methode zum entfernen von unerwünschtem, statischen Rauschen ist die Bipolar-Recording-Technique, bei der zwei Elektroden benutzt werden und die Differenz der beiden Signalen gebildet wird \citep{Day.2000}. Hierdurch können Störungen, die bei beiden Elektroden auftreten wie z.B. Hintergrundrauschen durch A/C-Stromversorgung entfernt werden. Wird jedoch ein Muskel in der Nähe einer der beiden Elektroden angeregt, so ist das Signal bei der näheren der beiden Elektroden viel stärker als bei der anderen, das Signal wird also nicht entfernt.\\

Der zweite wichtige Faktor für die Qualität der EMG-Messung ist die Impedanz zwischen Haut und Elektroden. Dank moderner Vorverstärker ist es nicht mehr wichtig, wie groß die Impedanz an den beiden Elektorden ist, sondern eine zeitliche Stabilität und Balance zwischen den beiden Elektroden \citep{Hermens.1999}. Da eine bipolare Messung nur Signalanteile, die bei beiden Elektroden gleich Auftreten entfernen kann, benötigt diese Methode möglichst ähnliche Bedinungen an beiden Elektroden. Aus dem selben Grund ist auch eine zeitliche Stabilität von hoher Wichtigkeit für eine verlässliche Messung. \\

Ein weiter Qualitätsfaktor ist das Überlagern von mehreren EMG-Signalen verschiedener Muskeln, auch Cross Talk genannt. Die Oberfläche-Elekroden messen naturgemäß nicht einen expliziten Muskel, sondern das elektrische Potential an der Hautoberfläche. Da die Stärke eines EMG-Signals proportional zu der Muskelgröße ist, können größere Muskeln Kleine überdecken. Dazu sinkt die Signalstärke exponentiell mit der Entfernung von der Elektrode, weshalb es schwer ist, tiefer liegende Muskeln zu messen. Diese Probleme können jedoch durch eine korrekte Elektrodengröße und -positionierung gelöst werden. \\

In welchem Zusammenhang ein EMG-Signal mit der resultierenden Muskelkraft steht ist noch nicht abschließend erforscht, wobei eine Vielzahl von Autoren eine Nicht-Lineare Proportionalität vermuten \citep{Karlsson.2001,Gregor.2002,Bilodeau.2003}. Dies hängt auch mit den oben erläuterten Problemen der EMG-Messung zusammen, weshalb es schwer ist, den Anteil einzelner Muskeln an dem Gesamtsignal zu bestimmen. Theoretische Analysen haben ergeben, dass die Amplitude des Signals bei isometrischen Kontraktionen mit der Quadratwurzel der generierten Kraft steigen sollte \citep{Uliam.2012,Basmajian.1985}. \\

\section{Signalverarbeitung}

TODO: Benötigte Filter etc. 

\section{Myoelektrische Steuerung}
Nach der Merkmalsextraktion wird eine Steuerungmechanismus benötigt, der basierend auf dem aufgenommenen Signal die Antriebseinheit steuert. Da es sich hierbei um einen Open-Loop-Kreis, also keine Rückkopplung stattfindet, handelt, spricht man von einer Steuerung, keiner Regelung. Abhängig vom erzielten Signal bieten sich hier drei Steuermechanismen an: Ein binärer Treshold-Mechanismus, bei der die Antriebseinheit mit einem fixen Drehmoment betrieben wird, sobald eine gewisse Schwelle übertreten wurde, ein Stepper-Mechanismus, bei dem Treshold-Mechanismen kombiniert werden, um mehrere diskrete Drehmomente zu ermöglichen. Oder eine proportionale Steuerung, bei der das Drehmoment direkt proportional zum Signal steigt. Welcher dieser drei Mechanismen realisierbar ist, hängt vorwiegend von der Qualität des aufgenommenen Signals ab. \\

Vorteile einer myoelektrischen Steuerung im Vergleich zu Joystick oder Drucksensoren sind unter andrem die Möglichkeit, die mechanische Unterstützung eines Exoskeletts problemlos einzustellen \citep{Ferris.2007},  das Signal durch Oberflächenelektroden nicht-invasiv aufgenommen werden kann und die benötigte Muskelrestaktivität verhältnismäßig klein ist \citep{Parker.2006} und einer intakten Gliedmaße ähnelt. Zusätzlich kann die Qualität der Steuerung durch neue Signalverarbeitungs-Algorithmen oder verbesserte Elektroden einfach und ohne medizinische Eingriffe verbessert werden und das Produkt somit langfristig im Einsatz bleiben. Auf der anderen Seite benötigt das System eine gewisse Einarbeitungszeit durch Patient und Therapeuten, bis die entsprechenden Parameter zur Steuerung angepasst sind. Da das aufgegriffene EMG-Signal stark von der Positionierung der Elektroden abhängt ist es ebenfalls wichtig, diese Position möglichst konstant zu halten. Ein weiterer Nachteil liegt darin, dass eine myoelektrische Steuerung naturgemäß nur möglich ist, wenn EMG-Signale vorliegen. Dadurch ist ein Einsatz bei einigen Krankheiten, z.B. Querschnittslähmung, nicht möglich. \\


\addcontentsline{toc}{chapter}{Bibliography}
\bibliography{./Literatur}
\bibliographystyle{apalike}

%Include SOPs (comment out/in as needed).
%\labday{Standard Operating Procedures}
%\inputSOPs

%%%%%%%%%%%%%%%%%%%%%%%%%%%%%%%%%%%%%%%%%%%%%%%%%%%%%%%%%%%%%%%%%%%%%%%%%%%%%%%%%%%%%%%%%%%%%%%%%%%%%%%%
%-------------------------------------------------------------------------------------------------------
\end{document}
%-------------------------------------------------------------------------------------------------------
%%%%%%%%%%%%%%%%%%%%%%%%%%%%%%%%%%%%%%%%%%%%%%%%%%%%%%%%%%%%%%%%%%%%%%%%%%%%%%%%%%%%%%%%%%%%%%%%%%%%%%%%