%%%%%%%%%%%%%%%%%%%%%%%%%%%%%%%%%%%%%%%%%%%%%%%%%%%%%%%%%%%%%%%%%%%%%%%%%%%%%%%%%%%%%%%%%%%%%%%%%%%%%%%%
%Packages for LabBook template:
%%%%%%%%%%%%%%%%%%%%%%%%%%%%%%%%%%%%%%%%%%%%%%%%%%%%%%%%%%%%%%%%%%%%%%%%%%%%%%%%%%%%%%%%%%%%%%%%%%%%%%%%
%
% Version: 02
%
% Changelist v02:
% ---------------------------------------------------------
% - FontSpec package removed, main font changed to 'Arial'.
%   Does not have to be compiled with XeLaTeX anymore.
% - Minor changes to comments.
% - Addition of svg package.
% - tabularx package removed since column width can be con-
%   trolled by adding "@{\extracolsep{\fill}}" before defining
%		columns (see example table for details).
%%%%%%%%%%%%%%%%%%%%%%%%%%%%%%%%%%%%%%%%%%%%%%%%%%%%%%%%%%%%%%%%%%%%%%%%%%%%%%%%%%%%%%%%%%%%%%%%%%%%%%%%

%Packages for character encoding and hyphenation.
\usepackage[T1]{fontenc}
\usepackage[english]{babel}

%Packages for graphics import.
\usepackage{graphicx,svg}

%Caption package to control the typesetting of float captions.
\usepackage[
labelfont={bf},
font={small},
singlelinecheck=false,
format=plain
]{caption}

%Standard AMS math package for math formatting
\usepackage{amsmath,amssymb,wasysym}

%SIunitx package for formatting of SI units. For options, see doc. Custom units are included.
\usepackage{siunitx}
\AtBeginDocument{\sisetup{per-mode=reciprocal, list-final-separator={, and }, range-phrase={\text{--}}, sticky-per, number-unit-product=\text{\,}, product-units=power, list-units=single, range-units=single, quotient-mode=fraction, fraction-function=\nicefrac, retain-unity-mantissa=false,math-rm=\mathrm, text-rm=\rmfamily, detect-weight, binary-units=true}}
\DeclareSIUnit\molar{M}
\DeclareSIUnit\particles{particles}
\DeclareSIUnit\rpm{rpm}
\DeclareSIUnit\units{\textsc{u}}
\DeclareSIUnit\ccm{ccm}
\DeclareSIUnit\psi{psi}
\DeclareSIUnit\rad{rad}
\DeclareSIUnit\fps{fps}
\DeclareSIUnit\sq{\ensuremath{\Box}}

%Macro to define function for times. I.e. \hms{1} gives '1 h', \hms{2;30} gives '2 h 30 min', \hms{;;30} gives '30 sec', etc...
\ExplSyntaxOn
\NewDocumentCommand \hms { o > { \SplitArgument { 2 } { ; } } m }
  {
    \group_begin:
      \IfNoValueF {#1}
        { \keys_set:nn { siunitx } {#1} }
      \siunitx_hms_output:nnn #2
    \group_end:
  }
\cs_new_protected:Npn \siunitx_hms_output:nnn #1#2#3
  {
    \IfNoValueF {#1}
      {
        \tl_if_blank:nF {#1}
          {
            \SI {#1} { \hour }
            \IfNoValueF {#2} { ~ }
          }
      }
    \IfNoValueF {#2}
      {
        \tl_if_blank:nF {#2}
          {
            \SI {#2} { \minute }
            \IfNoValueF {#3} { ~ }
          }
      }
    \IfNoValueF {#3}
      { \tl_if_blank:nF {#3} { \SI {#3} { \second } } }
  }
\ExplSyntaxOff

%Setspace package for line spacing macros.
\usepackage{setspace}

%Booktabs package for horizontal rules (\toprule, \midrule, \bottomrule).
\usepackage{booktabs}

%Scrlayer-scrpage package for formatting of header and footer.
\usepackage[headsepline=true,plainheadsepline=true,%
						footsepline=true,plainfootsepline=true]%
					 {scrlayer-scrpage}

%Package to input chemical formulae easier (\ce{H20}, \ce{H2S04}, \ce{Ca^2+}, etc.).
\usepackage[version=3]{mhchem}

%For hyperlinks between chapters, citations, etc.
\usepackage[plainpages=false,hidelinks]{hyperref}

%mathpazo package to change math font.
\usepackage{mathpazo}

%enumitem for control over list formatting.
\usepackage{enumitem}

%xcolor package for definition of colors etc.
\usepackage{xcolor}

%Code formatting
\usepackage{listings} %Library to list the code

%Tikz package for drawing line graphs etc.
\usepackage{tikz}
\usetikzlibrary{positioning} %library for relative alignments

%Circuitikz package for drawing circuits.
\usepackage{xstring}
\usepackage[europeanresistor]{circuitikz}

%grffile package helps with special characters in file and directory names.
\usepackage{grffile}

%geometry package allows flexible formatting of page margins etc.
\usepackage[vcentering]{geometry}

%lipsum package for blind text.
\usepackage{lipsum}

%Define a command to read in all .tex files from the individual years.
\def\inputLabDays{%
	\immediate\write18{cmd /c FileList_LabDays.bat}%
  \InputIfFileExists{./OutList.tmp}{}}

%Define a command to read in all .tex files from the SOP directory.
\def\inputSOPs{%
	\immediate\write18{cmd /c FileList_SOPs.bat}%
  \InputIfFileExists{./SOPs/OutList.tmp}{}}
	
%Scrhack package to fix \float@addtolist warning.
\usepackage{scrhack}
